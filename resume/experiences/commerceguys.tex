%---------------------------------------------------------
\cventry
{} % job title
{Commerce Guys} % organization
{Paris, France} % location
{} % date(s)
{
\begin{cvsubentries}
    \cvsubentry{référent technique France - cdi}{avril 2014 - décembre 2014}
    \begin{cvitems} % description(s) of tasks/responsibilities
        \item
        {
            Référent technique France de l'équipe service professionnel avec en charge 4
            développeurs drupal/drupal commerce. Etudes de développement, audit de sécurité et
            développement de modules pour de nombreux projets clients (total cso, niwel, bpost).
        }
        \item
        {
            Mission significative: développement d'un outil de migration et propagation de contenus
            sur différents sites de la Poste Belge (bpost) à partir d'une interface centralisée.
        }
    \end{cvitems}
    \cvsubentry{développer drupal back-end - cdi}{avril 2012 - mars 2014}
    \begin{cvitems} % description(s) of tasks/responsibilities
        \item
        {
            Etudes de développement, accompagnement technique, développements propriétaire et de modules
            pour de nombreux projets clients (total e-bouteilles, niwel, dior dam, mobistar, paypal, 
            la poste, kenzo, domyos). Développements significatifs: proxy rest avec symfony 2 et 
            requêtes soap, modules pour le tunnel d'achat e-commerce (livraison, moyen de paiement, promotions).
        }
        \item
        {
            Développement open source de nombreux modules drupal/drupal commerce dont les plus importants sont
            commerce rma, commerce discount, inline conditions. Ces deux derniers sont installés sur plus 11.000
            sites (donnée du 19/04/2020).
        }
        \item
        {
            Formateur drupal commerce pour de nombreuses ssii (niji, sqli et capgemini).
        }
    \end{cvitems}
\end{cvsubentries}
}
