%---------------------------------------------------------
\cventry
{} % job title
{commerce guys} % organization
{paris, france} % location
{} % date(s)
{
\begin{cvsubentries}
    \cvsubentry{référent technique france - cdi}{avril 2014 - décembre 2014}
    \begin{cvitems} % description(s) of tasks/responsibilities
        \item
        {
        référent technique france de l'équipe service professionel avec en charge 4
        développeurs drupal/drupal commerce. etudes de développement, audit de sécurité et
        développement de modules pour de nombreux projets clients (total cso, niwel, bpost).
        }
        \item
        {
        mission significative: développement d'un outil de migration et propagation de contenus
        sur différents sites de la poste belge (bpost) à partir d'une interface centralisée.
        }
    \end{cvitems}
    \cvsubentry{développer drupal back-end - cdi}{avril 2012 - mars 2014}
    \begin{cvitems} % description(s) of tasks/responsibilities
        \item
        {
        etudes de développement, accompagnement technique, développements propriétaire et de modules
        pour de nombreux projets clients (total e-bouteilles, niwel, dior dam, mobistar, paypal, 
        la poste, kenzo, domyos). développements significatifs: proxy rest avec symfony 2 et 
        requêtes soap, modules pour le tunnel d'achat e-commerce (livraison, moyen de paiement, promotions).
        }
        \item
        {
        développement open source de nombreux modules drupal/drupal commerce dont les plus importants sont
        commerce rma, commerce discount, inline conditions. ces deux derniers sont installés sur plus 11.000
        sites (donnée du 19/04/2020).
        }
        \item
        {
        formateur drupal commerce pour de nombreuses ssii (niji, sqli et capgemini).
        }
    \end{cvitems}
\end{cvsubentries}
}